\newpage
\begin{theorem}\label{thm:regret_ucb_alp_new}
Given $\pi_j$'s, $u_{j,k}$'s and a fixed $\rho \in [0,1]$, the regret of UCB-ALP satisfies:\\
1) (Non-boundary cases) if $\rho \neq q_j$, $1 \leq j \leq J-1$, then the regret of UCB-ALP is $R_{\rm{UCB-ALP}}(T, B) = O\big(JK\log T\big)$.\\
%\begin{eqnarray}
%\limsup_{T\to \infty} \frac{R_{\rm{UCB-ALP}}(T, B)}{\log T} \leq \Theta^{\rm (c)}_{\rm nb} + \Theta^{\rm (a)}, \nonumber
%\end{eqnarray}
%where $\Theta^{\rm (c)}_{\rm nb} = [\bar{u}^* + v(\rho)]\bigg\{\sum_{j = 1}^{\tilde{j}(\rho)}\sum_{k=1}^K\frac{27}{2g_{\tilde{j}(\rho)+1} [\Delta^{(j)}_{\tilde{j}(\rho)+1,k}]^2}
%+ \sum_{j =\tilde{j}(\rho)+2}^{J}\sum_{k=1}^K\frac{27}{2g_j [\Delta^{(\tilde{j}(\rho)+1)}_{j,k}]^2} + KJ \bigg\}$,
%$\Theta^{\rm (a)} = \sum_{j=1}^J \sum_{k \neq k_j^*} \big(\frac{2}{\Delta_{j,k}^{(j)}} + 2\Delta_{j,k}^{(j)}\big)$
%with $\bar{u}^* = \sum_{j = 1}^J \pi_j u_j^*$, and $v(\rho)$ defined in \eqref{eq:LP_single_step_value}.\\
2) (Boundary cases) if $\rho = q_j$ for some $j \in \{ 1,2, \ldots, J-1\}$, then the regret of UCB-ALP is $R_{\rm{UCB-ALP}}(T, B) = O\big(\sqrt{T} + JK\log T \big)$.
%\begin{eqnarray}
%R_{\text{UCB-ALP}}(T, B) \leq \Theta^{\rm (o)} \sqrt{T}  + [\Theta^{\rm (c)}_{\rm b} + \Theta^{\rm (a)}]\log T + O(1), \nonumber
%\end{eqnarray}
%where $\Theta^{\rm (o)} =  2(u_1^* - u_J^*) \sqrt{\rho (1-\rho)}$,
%%$\Theta^{\rm (a)}$ is defined as in Theorem~\ref{thm:regret_ucb_alp_nonboundary},
%\begin{align}
%\Theta^{\rm (c)}_{\rm b} = [\bar{u}^* + v(\rho)]\bigg\{\sum_{j = 1}^{\tilde{j}(\rho)-1}\sum_{k=1}^K\frac{27}{2g'_{\tilde{j}(\rho)} [\Delta^{(j)}_{\tilde{j}(\rho),k}]^2}
%+ \sum_{j =\tilde{j}(\rho)+1}^{J}\sum_{k=1}^K\frac{27}{2g'_j [\Delta^{(\tilde{j}(\rho))}_{j,k}]^2} + KJ \bigg\},
%\end{align}
%and $g'_j = \min\big\{\pi_j, \frac{1}{2}(\rho - q_{\tilde{j}(\rho)-1}), \frac{1}{2}(q_{\tilde{j}(\rho)+1} -\rho)\big\}$.
\end{theorem}


\begin{proof}
We bound the regret of UCB-ALP by comparing its performance  with the benchmark $\widehat{U}(T,B)$. To obtain this upper bound, we first partition the regret according to the sources and then bound the each part of regret, respectively.

Before presenting the proof, we first introduce a notation that will be widely used later. For contexts $j$ and $j'$, and an action $k$, let $\Delta_{j,k}^{(j')}$ be the difference between the expected reward for action $k$ under context $j$ and the highest expected reward under context $j'$, i.e., $\Delta_{j,k}^{(j')} = u_{j'}^* - u_{j,k}$.
When $j' = j$, $\Delta_{j,k}^{(j)}$ is the difference of expected reward between the suboptimal action $k$ and the best action under context $j$.


%%%%%%%%%%
\subsection{Step 1: Partition the Regret}
%%%%%%%%%%

Note that the total reward of the oracle solution $U^*(T,B)  \leq \widehat{U}(T,B)$. Thus, we can bound the regret of UCB-ALP by comparing its total expected reward $U_{\rm UCB-ALP}(T,B)$ with $\widehat{U}(T,B)$, i.e.,
\begin{eqnarray}
&&R_{\rm {UCB-ALP}}(T, B) \nonumber \\
&=& U^*(T,B) - U_{\rm UCB-ALP}(T,B) \nonumber \\
& \leq &\widehat{U}(T,B) - U_{\rm UCB-ALP}(T,B) \nonumber\\
&= & T v(\rho) - \sum_{j = 1}^J \sum_{k =1}^K u_{j,k} \mathbb{E}[C_{j,k}(T)].
\end{eqnarray}
The total expected reward of UCB-ALP can be further divided as
\begin{eqnarray}
&& U_{\rm UCB-ALP}(T,B) \nonumber \\
& =&  \sum_{j = 1}^J u_j^* \mathbb{E}\big[\sum_{k=1}^K C_{j,k}(T)\big] - \sum_{j=1}^J \sum_{k=1}^K \Delta_{j,k}^{(j)}\mathbb{E}[C_{j,k}(T)] \nonumber \\
& =&  \sum_{j = 1}^J u_j^*\mathbb{E}\big[C_{j}(T)\big] - \sum_{j=1}^J \sum_{k=1}^K \Delta_{j,k}^{(j)}\mathbb{E}[C_{j,k}(T)], \nonumber
\end{eqnarray}
where $C_j(T) = \sum_{k=1}^K C_{j,k}(T)$ is the total number that actions have been taken under context $j$ up to round $T$.

Consequently, the regret of UCB-ALP can be bounded as
\begin{eqnarray}\label{eq:regret_ucb_alp__nonboundary_details}
R_{\rm {UCB-ALP}}(T, B) \leq R_{\rm {UCB-ALP}}^{({\rm a})}(T, B) + R_{\rm {UCB-ALP}}^{({\rm c})}(T, B),
\end{eqnarray}
where
\begin{eqnarray}
R_{\rm {UCB-ALP}}^{({\rm a})}(T, B) = \sum_{j=1}^J \sum_{k=1}^K \Delta_{j,k}^{(j)}\mathbb{E}[C_{j,k}(T)], \nonumber
\end{eqnarray}
and
\begin{eqnarray}
R_{\rm {UCB-ALP}}^{({\rm c})}(T, B) = T v(\rho) - \sum_{j  = 1}^J u_j^*\mathbb{E}[C_j(T)]. \nonumber
\end{eqnarray}

Eq.~\eqref{eq:regret_ucb_alp__nonboundary_details} clearly shows that the regret of the UCB-ALP algorithm can be divided into two parts: the first part $R_{\rm {UCB-ALP}}^{({\rm a})}(T, B)$ is from taking  suboptimal actions under a given context; the second part $R_{\rm {UCB-ALP}}^{({\rm c})}(T, B)$ is from the deviation of remaining budget $b_{\tau}$ and context ranking errors.

%%%%%%%%%%
\subsection{Step 2: Bound Each Part of Regret}
%%%%%%%%%%

\subsubsection{Step 2.1: Bound of $R_{\text{UCB-ALP}}^{\rm (a)}(T, B)$.}
For the regret from action ranking errors,
we show in Lemma~\ref{thm:regret_withincontext} that $R_{\text{UCB-ALP}}^{\rm (a)}(T, B) = O(\log T)$ using similar techniques for traditional UCB methods \cite{Golovin2009Lecture}.
\begin{lemma}\label{thm:regret_withincontext}
Under UCB-ALP, the regret due to the action ranking errors within context $j$ satisfies
\begin{eqnarray} \label{eq:regret_ucb_alp_regret_withincontext}
R_{\text{UCB-PB}}^{\rm (a)}(T, B) \leq \sum_{j=1}^J \sum_{k \neq k_j^*} \bigg[\big(\frac{2}{\Delta_{j,k}^{(j)}} + 2\Delta_{j,k}^{(j)}\big) \log T + 2 \Delta_{j,k}^{(j)}\bigg].
\end{eqnarray}
\end{lemma}
\begin{proof}
For $k \neq k_j^*$, let $\ell_{j,k}^{(j)} = \frac{2\log T}{(\Delta_{j, k}^{(j)})^2}$. According to Lemma~\ref{thm:context_action_pair_errorprob}, we have
\begin{eqnarray}
&& \mathbb{E}[C_{j,k}(T)] \nonumber \\
&\leq& \ell_{j,k}^{(j)} + \sum_{t = 1}^T \mathbb{P}\{X_t = j, A_t = k, C_{j,k}(t-1) \geq \ell_{j,k}^{(j)}\} \nonumber \\
&\leq & \ell_{j,k}^{(j)} + \sum_{t = 1}^T \mathbb{P}\{X_t = j, A_t = k | C_{j,k}(t-1) \geq \ell_{j,k}^{(j)}\} \nonumber \\
&\leq & \ell_{j,k}^{(j)} + 2\sum_{t = 1}^Tt^{-1}. \nonumber
\end{eqnarray}
The conclusion then follows by the facts that $\sum_{t = 1}^T t^{-1} \leq 1 + \log T$ and $R_{\text{UCB-ALP}}^{\rm (a)}(T, B) =  \sum_{j = 1}^J\sum_{k \neq k_j^*} \Delta_{j,k}^{(j)} \mathbb{E}[C_{j,k}(T)]$.
\end{proof}

\subsubsection{Step 2.2: Bound of $R_{\text{UCB-ALP}}^{\rm (c)}(T, B)$.}
Next, we show that the second part $R_{\rm {UCB-ALP}}^{({\rm c})}(T, B)$ is also in the order of $O(\log T)$.
We mainly focus on the proof for the non-boundary cases, and discuss the boundary cases later.

From Algorithm~\ref{alg:ucb_alp}, we can see that the evolution of the remaining budget also affects the execution of the UCB-ALP algorithm. Under the assumption of known context distribution, it can be verified that Lemma~\ref{thm:alp_to_hypergeo} holds under UCB-ALP, i.e., the remaining budget $b_{\tau}$ follows the hypergeometric distribution and has the properties described in Lemma~\ref{thm:alp_to_hypergeo}. We define an event $\mathcal{E}_{\rm budget, 0}(t)$ as follows,
\begin{eqnarray}
\mathcal{E}_{\rm budget, 0}(t)  = \{(\rho - \delta) \tau \leq b_{\tau} \leq (\rho + \delta) \tau\}, \nonumber
\end{eqnarray}
where $\delta$ is given by
\begin{eqnarray}
\delta = \frac{1}{2}\min\{\rho - q_{\tilde{j}(\rho)}, q_{\tilde{j}(\rho)+1} -\rho\}. \nonumber
\end{eqnarray}
According to Lemma~\ref{thm:alp_to_hypergeo}, we have
\begin{eqnarray}
\mathbb{P}\{\urcorner\mathcal{E}_{\rm budget, 0}(t)\}
= \mathbb{P}\{ b_{\tau} <  (\rho -\delta) \tau\} + \mathbb{P}\{ b_{\tau} >  (\rho +\delta) \tau\} \leq 2e^{-2\delta^2 (T-t + 1)}. \nonumber
\end{eqnarray}

Now we analyze $\mathbb{E}[C_j(T)]$ under UCB-ALP. We first show the upper bound on $\mathbb{E}[C_j(T)]$ for $j \geq \tilde{j}(\rho) + 1$. Then, using the fact of $\sum_{j = 1}^J C_j(T) = B$, we can obtain the lower bound on $\mathbb{E}[C_j(T)]$ for $j \leq \tilde{j}(\rho) + 1$ and the upper bound of $R_{\text{UCB-ALP}}^{\rm (c)}(T, B)$.

For $j > \tilde{j}(\rho) + 1$ and $1 \leq k \leq K$, let $\ell_{j,k}^{(\tilde{j}(\rho)+1)} = \frac{2 \log T}{[\Delta_{j,k}^{(\tilde{j}(\rho)+1)}]^2}$. We know that if $(\rho - \delta) \tau \leq b_{\tau} \leq (\rho + \delta) \tau$, context $j$ can only be played when there exists a $j' \leq \tilde{j}(\rho)+1$ such that $\hat{u}_j(t) \geq \hat{u}_{j'}(t)$. Then, with a slight modification to the analysis of Lemma~\ref{thm:regret_withincontext}, we have
\begin{eqnarray}
\mathbb{E}[C_{j,k}(T)] &\leq& \ell_{j,k}^{(\tilde{j}(\rho)+1)} + \sum_{t = 1}^T \mathbb{P}\{X_t = j, \exists wrong-rank, C_{j,k}(t-1) \geq \ell_{j,k}^{(\tilde{j}(\rho)+1)}, \mathcal{E}_{\rm budget, 0}(t)\} + \sum_{t = 1}^T \mathbb{P}\{\urcorner\mathcal{E}_{\rm budget, 0}(t)\} \nonumber \\
&\leq&\ell_{j,k}^{(\tilde{j}(\rho)+1)} + 2(1 + \log T) + \frac{2e^{-2\delta^2}}{1 - e^{-2\delta^2}}, \nonumber
\end{eqnarray}
implying that
\begin{eqnarray} \label{eq:ub_supoptimal_context}
\mathbb{E}[C_j(T)] \leq \sum_{k=1}^K \ell_{j,k}^{(\tilde{j}(\rho)+1)} + 2K\big(1 + \log T + \frac{2e^{-2\delta^2}}{1 - e^{-2\delta^2}}\big).
\end{eqnarray}


For $j = \tilde{j}(\rho) + 1$, we have
\begin{eqnarray}
\mathbb{E}[C_{j,k}(T)] &\leq& \ell_{\tilde{j}(\rho)+1,k}^{(j)} + \sum_{t = 1}^T \mathbb{P}\{X_t = j, \exists wrong-rank, C_{j,k}(t-1) \geq \ell_{j,k}^{(\tilde{j}(\rho)+1)}, \mathcal{E}_{\rm budget, 0}(t)\} \nonumber \\
&& + \sum_{t = 1}^T \mathbb{P}\{X_t = j, no wrong-rank, C_{j,k}(t-1) \geq \ell_{j,k}^{(\tilde{j}(\rho)+1)}, \mathcal{E}_{\rm budget, 0}(t)\} + \sum_{t = 1}^T \mathbb{P}\{\urcorner\mathcal{E}_{\rm budget, 0}(t)\} \nonumber \\
&\leq&\ell_{\tilde{j}(\rho)+1,k}^{(j)} + 2(1 + \log T)  + \frac{2e^{-2\delta^2}}{1 - e^{-2\delta^2}}, \nonumber
\end{eqnarray}





\end{proof} 

\section{General UCB-ALP}

How to show its logarithmic regret?

Analyze the scenarios where a certain context-action pair can be pulled:

1) $\hat{u}_{j,k}/c_{j,k}$ is large;

2) $(\hat{u}_{j,k_1} - \hat{u}_{j,k_2})/(c_{j, k_1} - c_{j, k_2})$ is larger, then $\hat{u}_{j,k_1}$ will be pulled. Moreover, it will be greater than 0 with high probability $O(1/T)$ if $(j,k_1)$ has been pulled for sufficient times.
\begin{lemma} \label{thm:hetcost_ucb_rank}
1) If $u_{j_1,k_1}/c_{j_1,k_1} <  u_{j_2,k_2}/c_{j_2,k_2}$, then $\mathbb{P}\{\hat{u}_{j_1,k_1}(t)/c_{j_1,k_1} \geq   \hat{u}_{j_2,k_2}(t)/c_{j_2,k_2}| C_{j_1, k_1}(t-1) \geq ?? \} \leq t^{-1}$.

2) For contexts $j_1$, $j_2$, and actions $k_{11}$, $k_{12}$, $k_{21}$, and $k_{22}$, if $u_{j_1,k_{11}} > u_{j_1,k_{12}}$, $u_{j_2,k_{21}} > u_{j_2,k_{22}}$, and $\frac{u_{j_1,k_{11}} - u_{j_1,k_{12}}}{c_{j_1,k_{11}} - c_{j_1,k_{12}}} <  \frac{u_{j_2,k_{21}} - u_{j_2,k_{22}}}{c_{j_2,k_{21}} - c_{j_2,k_{22}}}$, then
\begin{equation}
\mathbb{P}\big\{ \frac{u_{j_1,k_{11}}(t) - u_{j_1,k_{12}}(t)}{c_{j_1,k_{11}} - c_{j_1,k_{12}}} \geq  \frac{u_{j_2,k_{21}}(t) - u_{j_2,k_{22}}(t)}{c_{j_2,k_{21}} - c_{j_2,k_{22}}} \big| C_{j_1, k_{11}}(t-1) \geq \red{TBD} ?? , C_{j_2, k_{22}}(t-1) \geq  \red{TBD}\big\} \leq \frac{1}{t}.
\end{equation}
\end{lemma}

\textbf{The maxim of ``Optimism in the Face of Uncertainty''}: being optimistic within the confidence range; if one context-action should not be pulled that much, it will be finally identified (as soon as it has been pulled sufficiently)!

\textbf{The key idea of proof:} a context-action pair $(j,k)$ could only be pulled when it has large UCB. If it has been pulled for enough times, its UCB will be relatively accurate and will not be over pulled.

According to Lemma~\ref{thm:hetcost_ucb_rank}, we know that there exists a positive number $\delta_u$ such that if $|\hat{u}_{j,k}(t) - u_{j,k}| \leq \delta_u$, then the UCB-ALP will provide a correct rank (and thus \blue{a plausible solution}). 

Furthermore, we only need to consider the cases where there will actually result in regret and only consider  the context-action pairs with positive probability.
Let $\hat{C}_{j,k}(t)$ be the number of slots where $\hat{p}_{j,k}(t') > \delta_p > 0$, but $|\hat{u}_{j,k}(t') - u_{j,k}| \geq \delta_u$ up to $t$, i.e.,
\begin{equation}
\hat{C}_{j,k}(t) = \sum_{t' = 1}^{t}\mathbbm{1}(\hat{p}_{j,k}(t') > \delta_p, |\hat{u}_{j,k}(t') - u_{j,k}| \geq \delta_u).
\end{equation}
Then we can bound $\hat{C}_{j,k}(t)$ as $\hat{C}_{j,k}(t)  = O(\log T)$. 

Note that a suboptimal solution will be implemented only when there exists a context-action pair $(j,k)$ such that $\hat{p}_{j,k}(t') > \delta_p > 0$ and $|\hat{u}_{j,k}(t') - u_{j,k}| \geq \delta_u$. Summarize over all context-actions pairs, we know that the total regret will be bounded as $O(\log T)$. 
