%%%%%%%%%%
\section{Proof of Theorem~\ref{thm:alp_regret}: Near Optimality of ALP}\label{app:proof_alp_regret}
%%%%%%%%%%

\subsection{Proof of Lemma~\ref{thm:alp_to_hypergeo}: Evolution of Remaining Budget}

The evolution of the remaining budget $b_\tau$ is critical  for evaluating the expected total reward under ALP. We prove Lemma~\ref{thm:alp_to_hypergeo} by casting ALP to a sampling problem without replacement.

From the implementation of ALP, we can verify that when the remaining time is $\tau$ and remaining budget is $b_{\tau} = b$,
the system consumes one unit of budget with probability $b/\tau$, and consumes nothing with probability $1-b/\tau$.
Thus, when focusing on the remaining budget,
we can view the ALP algorithm as a sampling problem without replacement as follows.

\textbf{Mapping ALP to Sampling without Replacement:} Consider $T$ balls in an urn,
including $B$ black balls and $T-B$ white balls. Running ALP is equivalent to randomly drawing a ball without replacement.
Taking an action $A_t > 0$ is equivalent to drawing  a black ball and taking the dummy action $A_t = 0$ is equivalent to drawing a white ball.
The event that $b_{\tau} = b$ is equivalent to the event that the agent draws $T- \tau$ balls, and the number of drawn black balls is $B-b$.

Based on the above mapping and using its symmetric property, we know that $b_\tau$ follows the hypergeometric distribution \cite{Dubhashi2009Concentration} and complete the proof of Lemma~\ref{thm:alp_to_hypergeo}.

\subsection{Part 1: Non-Boundary Cases}\label{app:proof_alp_regret_non_boundary}
According to Lemma~\ref{thm:upper_bound}, $\widehat{U}(T,B)$ is an upper bound on the total expected reward. Thus,
\begin{eqnarray} \label{eq:alp_regret}
U^*(T,B) - U_{\rm ALP}(T,B)
\leq  \widehat{U}(T,B) - U_{\rm ALP}(T,B) =  \sum_{\tau = 1}^T \big\{{v}(\rho)-\mathbb{E}[{v}(b_\tau/\tau)]\big\}.
\end{eqnarray}
To evaluate the gap between the single-round values, we define an auxiliary function $\tilde{v}(b/\tau)$ for a given $\rho$ as follows:
\begin{eqnarray}\label{eq:approx_single_round_alp}
\tilde{v}(b/\tau) = \sum_{j = 1}^{\tilde{j}(\rho) }\pi_j u_j^*
 +\pi_{\tilde{j}(\rho) + 1} \tilde{p}_{\tilde{j}(\rho) + 1}(b/\tau) u_{\tilde{j}(\rho) + 1}^*,
\end{eqnarray}
where
\begin{eqnarray}
\tilde{p}_{\tilde{j}(\rho) + 1}(b/\tau) = \frac{b/\tau - \sum_{j = 1}^{\tilde{j}(\rho)} \pi_j} {\pi_{\tilde{j}(\rho) +1}}. \nonumber
\end{eqnarray}
This auxiliary function bridges the gap of single-round values, $v(\rho)$ and $\mathbb{E}[v(b_\tau/\tau)]$, as follows:

First, we note that $\tilde{v}(b/\tau)$ uses the same threshold $\tilde{j}(\rho)$ as in $v(\rho)$.
The only difference between $\tilde{v}(b/\tau)$ and $v(\rho)$ is that $\tilde{v}(b/\tau)$ uses the instantaneous
average budget $b/\tau$ instead of the fixed average budget $\rho$. Considering all possible $b$'s and according to Lemma~\ref{thm:alp_to_hypergeo},
we have
\begin{eqnarray}\label{eq:virtual_and_upperbound}
{v}(\rho) - \mathbb{E}[{\tilde{v}}(b_{\tau}/\tau)] = \big\{\rho - \mathbb{E}[b_{\tau}/\tau]\big\}u_{\tilde{j}(\rho) + 1}^* = 0.
\end{eqnarray}

Second, compared with $v(b/\tau)$, the difference of the auxiliary function $\tilde{v}(b/\tau)$ comes
from the event of $\tilde{j}(b/\tau) \neq \tilde{j}(\rho)$, which only occurs when $b/\tau < q_{\tilde{j}(\rho)}$
or $b/\tau > q_{\tilde{j}(\rho)+1}$.
Because $\rho \neq q_j$, $1 \leq j \leq J-1$,  there exists a positive number $\delta = \min\{\rho -q_{\tilde{j}(\rho)}, q_{\tilde{j}(\rho)+1} - \rho\}$
%\footnote{For example, $\delta = \min\{q_{\tilde{j}(\rho)-1} - \rho, \rho -q_{\tilde{j}(\rho)}\}$ if $3 \leq \tilde{j}(\rho) \leq J$,
%$\delta = \rho - q_{\tilde{j}(\rho)}$ if $\tilde{j}(\rho) = 2$,
% $\delta = q_{\tilde{j}(\rho)-1} - \rho$ if $\tilde{j}(\rho) = J + 1$.}
 such that  for all $\rho - \delta \leq \rho' < \rho + \delta$, the threshold under $\rho'$ is the same as that under $\rho$, i.e.,
$
\tilde{j}(\rho') = \tilde{j}(\rho). \nonumber
$
Therefore, for all $b$ satisfying $\rho - \delta \leq b/\tau \leq \rho + \delta$,
$v(b/\tau) = \tilde{v}(b/\tau)$.
%\footnote{For $b/\tau = \min\{1, \rho + \delta\}$, the threshold may change but then the randomization value will zero.}.

If $b/\tau < \rho - \delta$, then
\begin{eqnarray} \label{eq:single_step_gap_left}
&&\tilde{v}(b/\tau) -  v(b/\tau) \nonumber \\
&=&  \bigg[ \sum_{j = \tilde{j}(b/\tau)  +1 }^{\tilde{j}(\rho)} \pi_j u_j^*
+ \big(\frac{b}{\tau} - q_{ \tilde{j}(\rho)}\big) u_{\tilde{j}(\rho) + 1}^*  \bigg]  - \big(\frac{b}{\tau} - q_{ \tilde{j}(b/\tau)}\big) u_{\tilde{j}(b/\tau)  +1}^* \nonumber \\
&\leq &  \bigg[ u_1^* \sum_{j = \tilde{j}(b/\tau)  +1 }^{\tilde{j}(\rho)} \pi_j
+ \big(\frac{b}{\tau} - q_{ \tilde{j}(\rho)}\big) u_{\tilde{j}(\rho) + 1}^*  \bigg]  - \big(\frac{b}{\tau} - q_{ \tilde{j}(b/\tau)   }\big) u_{\tilde{j}(\rho) + 1}^* \nonumber \\
&\leq& (u_1^* - u_{\tilde{j}(\rho) + 1}^*)\sum_{j = \tilde{j}(b/\tau)  +1 }^{\tilde{j}(\rho)} \pi_j     \nonumber \\
&\leq & q_{\tilde{j}(\rho)}(u_1^* -  u_J^*).
\end{eqnarray}

Similarly, if $b/\tau > \rho + \delta$, then
\begin{eqnarray} \label{eq:single_step_gap_right}
&& \tilde{v}(b/\tau) -  v(b/\tau) \nonumber \\
 &=&
\big(\frac{b}{\tau} - q_{ \tilde{j}(\rho)}\big) u_{\tilde{j}(\rho) +1}^* - \bigg[ \sum_{j = \tilde{j}(\rho)+1}^{\tilde{j}(b/\tau)} \pi_j u_j^* + \big(\frac{b}{\tau} - q_{ \tilde{j}(b/\tau) }\big) u_{\tilde{j}(b/\tau)+1}^*  \bigg] \nonumber \\
&\leq& (1 - q_{\tilde{j}(\rho)}) (u_1^* - u_J^*).
\end{eqnarray}

Summing all the above three cases ($\rho - \delta \leq b/\tau \leq \rho + \delta$, $b/\tau < \rho - \delta$, and $b/\tau > \rho + \delta$) and using Eq.~\eqref{eq:virtual_and_upperbound}, we have
\begin{eqnarray} \label{eq:virtual_and_alp}
&&v(\rho) - \mathbb{E}[v(b_\tau/\tau)] \nonumber\\
&=& v(\rho)- \mathbb{E}[\tilde{v}(b_\tau/\tau)] + \mathbb{E}[\tilde{v}(b_\tau/\tau) - v(b_\tau/\tau)]  \nonumber\\
& =&  \sum_{ b < \tau(\rho - \delta)~{\rm or}~b >\tau(\rho + \delta)} \mathbb{P}(b_\tau = b)[\tilde{v}(b/\tau) -v(b/\tau)] \nonumber\\
&\leq & q_{\tilde{j}(\rho)}(u_1^* - u_J^*) \mathbb{P}\{b_{\tau} < \tau(\rho - \delta)\} \nonumber \\
&&+ (1-q_{\tilde{j}(\rho)}) (u_1^* - u_J^*) \mathbb{P}\{b_{\tau} > \tau(\rho + \delta)\}     \nonumber \\
&\leq & (u_1^* - u_J^*) e^{-2\delta^2\tau}.
\end{eqnarray}

Part 1 of Theorem~\ref{thm:alp_regret} then follows by substituting Eq.~\eqref{eq:virtual_and_alp} into Eq.~\eqref{eq:alp_regret}.


\subsection{Part~2: Boundary Cases}\label{app:proof_alp_regret_all}
The proof of Part~2 of Theorem~\ref{thm:alp_regret} is similar to that of Part~1. Specifically, when  $\rho = q_{\tilde{j}(\rho)}$, let $\delta' = \min\{\rho - q_{\tilde{j}(\rho)-1}, q_{\tilde{j}(\rho)+1} - \rho\}$.   From the proof of Part 1,
we know that
 \begin{eqnarray}
v(\rho)- \mathbb{E}[\tilde{v}(b_\tau/\tau)] = 0.
\end{eqnarray}

In addition, if $\rho \leq b/\tau \leq \rho + \delta'$, then $\tilde{j}(b/\tau) =  \tilde{j}(\rho)$ and $v(b/\tau) = \tilde{v}(b/\tau)$.

If $\rho - \delta' \leq b/\tau <\rho$, we have $\tilde{j}(b/\tau) = \tilde{j}(\rho) - 1$, and
\begin{eqnarray}
&&\tilde{v}(b/\tau) - v(b/\tau) \nonumber \\
& = & \pi_{\tilde{j}(\rho)} u^*_{\tilde{j}(\rho)} + (\frac{b}{\tau} - q_{\tilde{j}(\rho)})u^*_{\tilde{j}(\rho)+ 1}  - (\frac{b}{\tau} - q_{\tilde{j}(b/\tau)})u^*_{\tilde{j}(b/\tau) + 1} \nonumber \\
& = & (\pi_{\tilde{j}(\rho)} + q_{\tilde{j}(\rho) - 1} - \frac{b}{\tau})u^*_{\tilde{j}(\rho)} + (\frac{b}{\tau} - q_{\tilde{j}(\rho)})u^*_{\tilde{j}(\rho)+ 1} \nonumber \\
& = & (\rho - \frac{b}{\tau})u^*_{\tilde{j}(\rho)} + (\frac{b}{\tau} - \rho)u^*_{\tilde{j}(\rho)+ 1} \nonumber \\
&\leq & \big|\frac{b}{\tau} - \rho\big|(u_1^* - u^*_J).
\end{eqnarray}

Moreover, we still have \eqref{eq:single_step_gap_left} if  $b/\tau < \rho - \delta'$, and \eqref{eq:single_step_gap_right} if $b/\tau > \rho + \delta'$.

Compared with the proof of Part~1, we know that the only difference relies on the case of $\rho - \delta' \leq b/\tau <\rho$. Thus, summing all the above cases and using the results in the analysis of Part~1, we have
\begin{eqnarray}
\mathbb{E}[\tilde{v}(b_\tau/\tau) - v(b_{\tau}/\tau)] \leq (u_1^* - u_J^*)\big\{\mathbb{E}[|b_{\tau}/\tau - \rho|] + e^{-2(\delta')^2\tau}\big\} \leq  (u_1^* - u_J^*)\big[\sqrt{\frac{{\rm Var}(b_{\tau})}{\tau^2}} + e^{-2(\delta')^2\tau}\big] . \nonumber
\end{eqnarray}
Consequently,
\begin{eqnarray}
 U^*(T,B) -  U_{\rm ALP}(T,B) &\leq & \widehat{U}(T,B) -  U_{\rm ALP}(T,B) \nonumber \\
&=&  \sum_{\tau = 1}^T \big\{v(\rho)-\mathbb{E}[v(b_\tau/\tau)]\} \nonumber \\
&=& \sum_{\tau = 1}^T \big\{v(\rho)- \mathbb{E}[\tilde{v}(b_\tau/\tau)] \big\} + \sum_{\tau = 1}^T  \mathbb{E}[ \tilde{v}(b_\tau/\tau) - v(b_{\tau}/\tau) ]  \nonumber\\
&\leq& (u_1^* - u_J^*)\sum_{\tau = 1}^T\big[\sqrt{\frac{{\rm Var}(b_{\tau})}{\tau^2}} + e^{-2(\delta')^2\tau}\big] \nonumber\\
& = &(u_1^* - u_J^*)\sum_{\tau = 1}^T\big[\sqrt{\frac{(T-\tau)\rho(1-\rho)}{(T-1)\tau}}+ e^{-2(\delta')^2\tau}\big]  \nonumber\\
& \leq &(u_1^* - u_J^*) \sqrt{\rho(1-\rho)} \sum_{\tau = 1}^T\sqrt{\frac{1}{\tau}}  + \frac{u_1^* - u_J^*}{1-  e^{-2(\delta')^2}}\nonumber\\
&\leq &2\sqrt{\rho(1-\rho)} (u_1^* - u_J^*) \sqrt{T} + \frac{u_1^* - u_J^*}{1- e^{-2(\delta')^2}}. \nonumber
\end{eqnarray}
